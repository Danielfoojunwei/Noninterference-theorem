\documentclass[11pt,a4paper]{article}

% ------------------------------------------------------------------
% Packages
% ------------------------------------------------------------------
\usepackage[utf8]{inputenc}
\usepackage[T1]{fontenc}
\usepackage{amsmath,amssymb,amsthm,mathtools}
\usepackage{hyperref}
\usepackage[numbers]{natbib}
\usepackage{enumitem}
\usepackage{booktabs}
\usepackage[margin=1in]{geometry}

% ------------------------------------------------------------------
% Theorem environments
% ------------------------------------------------------------------
\newtheorem{theorem}{Theorem}
\newtheorem{lemma}[theorem]{Lemma}
\newtheorem{corollary}[theorem]{Corollary}
\newtheorem{definition}[theorem]{Definition}
\newtheorem{remark}[theorem]{Remark}

% ------------------------------------------------------------------
% Convenience macros
% ------------------------------------------------------------------
\newcommand{\St}{S_t}
\newcommand{\Pt}{P_t}
\newcommand{\Mt}{M_t}
\newcommand{\Bt}{B_t}
\newcommand{\Gt}{G_t}
\newcommand{\Vt}{V_t}
\newcommand{\Et}{E_t}
\newcommand{\calT}{\mathcal{T}}
\newcommand{\calP}{\mathcal{P}}
\newcommand{\taint}{\tau}
\newcommand{\prov}{\pi}
\newcommand{\SYS}{\textsf{SYS}}
\newcommand{\USER}{\textsf{USER}}
\newcommand{\TOOL}{\textsf{TOOL}}
\newcommand{\TOOLauth}{\textsf{TOOL}_{\textsf{auth}}}
\newcommand{\TOOLunauth}{\textsf{TOOL}_{\textsf{unauth}}}
\newcommand{\WEB}{\textsf{WEB}}
\newcommand{\SKILL}{\textsf{SKILL}}

% ------------------------------------------------------------------
\title{Noninterference Theorem for Indirect Prompt Injection\\in Agentic AI}
\author{}
\date{}

\begin{document}
\maketitle

% ==================================================================
\begin{abstract}
Agentic AI systems routinely ingest untrusted content---emails, web pages and
documents---within the same context window that contains system prompts and
user instructions.  Indirect prompt injection exploits this shared context to
embed malicious instructions in otherwise legitimate data, silently influencing
the agent's behaviour.  We formalise the security requirement as a
\emph{noninterference} property adapted from information-flow
control~\cite{goguen1982}: adversarial variations in untrusted data must not
influence the agent's decisions or modify its control plane.  We model the
agent as a discrete-time dynamical system, define a typed intermediate
representation with taint tracking and a trust lattice, and prove that a
\textbf{system-level} enforcement layer---a deterministic verifier that filters
tainted content from the action-selection dependency set before each inference
step---guarantees action invariance to untrusted input variations.  The theorem
is a property of the \emph{enforcement architecture}, not of the underlying
language model's internal reasoning: it holds because tainted content is
excluded from the model's input when selecting actions, not because the model
chooses to ignore it.
\end{abstract}

% ==================================================================
\section{Introduction}\label{sec:intro}

Agentic AI systems autonomously execute tool calls and chain operations across
external services.  Because they ingest untrusted content---retrieved web
pages, user-uploaded documents, skill outputs---alongside system prompts and
direct user instructions, they are vulnerable to \emph{indirect prompt
injection}: an adversary embeds hidden instructions in external data that
reshape the agent's intent, redirect tool usage, or trigger unauthorised
actions~\cite{crowdstrike2025,esecplanet2025}.

CrowdStrike notes that indirect prompt injection allows adversaries to poison
the environment without directly interacting with the
agent~\cite{crowdstrike2025}.  eSecurity Planet further observes that current
systems do not enforce a hard separation between explicit user intent and
third-party content~\cite{esecplanet2025}, so information retrieved during a
task is processed in the same reasoning context as direct instructions.  Real
attacks have already been used to drain crypto wallets and exfiltrate private
channel messages~\cite{esecplanet2025}.

The fundamental security requirement is \textbf{noninterference}: adversarial
variations in untrusted data should not influence the agent's decisions or
modify its control plane.  In this paper we show how a system-level enforcement
layer---comprising taint tracking, a typed intermediate representation, and a
deterministic verifier---can guarantee this property.

\paragraph{Critical distinction.}
The theorem is about the \emph{system-level action-selection function}, not
about the language model's internal reasoning.  We do not claim the model
ignores injections; we prove that when tainted content is architecturally
excluded from the model's input at the action-selection step, the resulting
actions are invariant to untrusted input variations.

\paragraph{Contributions.}
\begin{enumerate}[label=(\roman*)]
  \item A formal model of agentic state incorporating a typed intermediate
        representation (IR) graph with taint tracking and a trust lattice
        that distinguishes authenticated from unauthenticated tool
        outputs (\S\ref{sec:model}).
  \item The \emph{Noninterference Theorem}: under stated invariants, the
        system-level action-selection function produces identical tool calls
        and control-plane updates when untrusted inputs vary, provided
        tainted content is excluded from the action dependency
        set (\S\ref{sec:theorem}).
  \item A discussion of the theorem's scope: it formalises a filtering rule,
        which we argue is valuable as a precise system specification, while
        acknowledging the utility gap and the need for a concrete
        declassification protocol (\S\ref{sec:discussion}).
  \item An empirical validation on InjecAgent~\cite{zhan2024injecagent}
        (1,054~test cases) and BIPIA~\cite{yi2023bipia} using real neural
        network inference with FLAN-T5-base as a surrogate decision-point
        model (\S\ref{sec:eval}).
\end{enumerate}

% ==================================================================
\section{Background: Indirect Prompt Injection as a Control-Plane Threat}
\label{sec:background}

Agentic AI systems routinely ingest untrusted content such as emails, web
pages and documents.  This data is mixed into the same context window that
contains system prompts and user instructions.  CrowdStrike notes that
indirect prompt injection allows adversaries to poison the environment without
directly interacting with the agent; malicious instructions embedded in
otherwise legitimate data can silently influence the agent's
behaviour~\cite{crowdstrike2025}.  Because agentic systems autonomously
execute tool calls and chain operations, untrusted data can reshape intent,
redirect tool usage and trigger unauthorised actions~\cite{crowdstrike2025}.

eSecurity Planet further observes that current systems do not enforce a hard
separation between explicit user intent and third-party
content~\cite{esecplanet2025}, so information retrieved during a task is
processed in the same reasoning context as direct
instructions~\cite{esecplanet2025}.  Attackers have already used indirect
injection to drain crypto wallets and to exfiltrate private channel
messages~\cite{esecplanet2025}.

Prior defenses operate at the prompt level and provide probabilistic protection:
border strings, sandwich prompting, and instructional defense
(Yi et al., 2023)~\cite{yi2023bipia} reduce ASR to 12--20\% on GPT-4 but
cannot eliminate it.  SpotLight~\cite{hines2024spotlight} achieves stronger
reductions (${>}50\% \to {<}2\%$ with encoding) but remains probabilistic:
the model retains access to injection content and may still follow it in some
settings.

% ==================================================================
\section{Model Definitions}\label{sec:model}

% definitions.tex -- Formal model definitions
% Included by main.tex in Section 3 (Model Definitions)

We model the agent as a discrete-time dynamical system that maintains
structured state and uses a recursive language model (RLM) to choose actions.

% ------------------------------------------------------------------
\subsection{State Variables}\label{sec:state}

Let the system state at step~$t$ be
\begin{equation}\label{eq:state}
  \St = (\Pt,\, \Mt,\, \Bt,\, \Gt),
\end{equation}
where:
\begin{itemize}
  \item $\Pt$ is the \emph{control-plane state} containing permissions, tool
    policies and integration configuration.
  \item $\Mt$ is a \emph{memory store} of verified facts and historical data.
  \item $\Bt$ is a \emph{risk budget} representing how much risk can be spent
    on tool calls in this session.
  \item $\Gt = (\Vt, \Et)$ is a \emph{typed intermediate representation (IR)
    graph} summarising the current context.
\end{itemize}

Each node $v \in \Vt$ has a type $\mathrm{type}(v)$ drawn from the set
\begin{equation}\label{eq:types}
  \calT = \bigl\{
    \textsf{Policy},\;
    \textsf{UserIntent},\;
    \textsf{TrustedConfig},\;
    \textsf{UntrustedQuote},\;
    \textsf{CandidateFact},\;
    \textsf{VerifiedFact},\;
    \textsf{ToolResult},\;
    \textsf{ActionRequest}
  \bigr\}.
\end{equation}

% ------------------------------------------------------------------
\subsection{Trust Lattice and Taint Labels}\label{sec:trust}

Inputs are associated with principals
\begin{equation}\label{eq:principals}
  \calP = \{\SYS,\; \USER,\; \TOOL,\; \WEB,\; \SKILL\},
\end{equation}
ordered by authority $\preceq$ so that untrusted sources satisfy
\[
  \WEB,\;\SKILL \;\preceq\; \TOOL \;\preceq\; \USER \;\preceq\; \SYS.
\]

\begin{definition}[Provenance and taint]\label{def:taint}
Each node~$v$ carries:
\begin{enumerate}[label=(\alph*)]
  \item a \emph{provenance label} $\prov(v) \in \calP$, recording which
    principal produced the data; and
  \item a \emph{taint bit} $\taint(v) \in \{0,1\}$, where $\taint(v) = 1$
    indicates that~$v$ was derived from untrusted input.
\end{enumerate}
Raw spans originating from $\WEB$ or $\SKILL$ are tainted ($\taint = 1$).
Facts extracted from tainted spans remain tainted until explicitly promoted to
$\textsf{VerifiedFact}$ by a verification procedure.
\end{definition}

\begin{definition}[Taint propagation]\label{def:taint-prop}
If node~$v$ depends on any node~$u$ with $\taint(u) = 1$, then
$\taint(v) = 1$.  Formally, for the dependency relation~$\to$ on~$\Vt$:
\[
  \taint(v) = \max_{u :\, u \to v} \taint(u).
\]
The only exception is the $\textsf{VerifiedFact}$ promotion rule: a dedicated
verification procedure may set $\taint(v) = 0$ after checking the content of
$v$ against trusted sources.
\end{definition}

% ------------------------------------------------------------------
\subsection{Update Function and Verifier}\label{sec:update}

At each step, a recursive controller $T_\theta$ proposes an updated state and
a candidate tool call:
\begin{equation}\label{eq:controller}
  (\hat{S}_{t+1},\, \hat{y}_t,\, \hat{a}_t) = T_\theta(\St,\, I_t),
\end{equation}
where $I_t$ denotes new inputs at step~$t$ (user message, retrieved data and
tool outputs).

\begin{definition}[Verifier]\label{def:verifier}
A deterministic verifier $V$ checks the following invariants on every proposed
transition:
\begin{enumerate}[label=(V\arabic*)]
  \item \label{inv:taint-dep}
    \textbf{Taint-free action dependence.}
    The candidate tool call~$\hat{a}_t$ depends only on IR nodes with
    $\taint(v) = 0$.
  \item \label{inv:cp-authority}
    \textbf{Control-plane authority.}
    Any proposed modification to $\Pt$ originates from a principal~$p$ with
    $p \succeq \USER$ (i.e.\ only $\SYS$ or~$\USER$ may alter the control
    plane).
  \item \label{inv:memory-authority}
    \textbf{Memory authority.}
    Entries added to $\Mt$ from tainted sources are marked as
    $\textsf{CandidateFact}$ and not as $\textsf{VerifiedFact}$.
  \item \label{inv:budget}
    \textbf{Risk budget.}
    The proposed action does not exceed the remaining risk budget~$\Bt$.
\end{enumerate}
\end{definition}

If the verifier accepts ($V = 1$), the proposed update is applied.  Otherwise,
a safe repair is applied:
\begin{equation}\label{eq:repair}
  (S_{t+1},\, y_t,\, a_t) =
  \begin{cases}
    (\hat{S}_{t+1},\, \hat{y}_t,\, \hat{a}_t),
      & \text{if } V = 1, \\[4pt]
    \bigl(\mathrm{Repair}(\St),\, y_t^{\mathrm{safe}},\, \varnothing\bigr),
      & \text{otherwise.}
  \end{cases}
\end{equation}

The $\mathrm{Repair}$ function resets any fields that would violate invariants,
emits a safe (no-op) output~$y_t^{\mathrm{safe}}$, and suppresses the tool
call ($a_t = \varnothing$).

\begin{remark}
The verifier is \emph{deterministic}: its output depends only on the proposed
transition and the current state, not on any stochastic component.  This is
essential for the noninterference proof, because it ensures the repair path
does not introduce dependence on untrusted data.
\end{remark}


% ==================================================================
\section{Noninterference Theorem}\label{sec:theorem}

% proof.tex -- Noninterference theorem statement and proof
% Included by main.tex in Section 4 (Noninterference Theorem)

We now state and prove the main result.

% ------------------------------------------------------------------
\subsection{Theorem Statement}

\begin{theorem}[Noninterference for Indirect Prompt Injection]
\label{thm:noninterference}
Consider two executions of the agent system, indexed by $i \in \{1,2\}$,
satisfying:
\begin{enumerate}[label=(A\arabic*)]
  \item \label{ass:init}
    Identical initial state: $S_0^{(1)} = S_0^{(2)}$.
  \item \label{ass:trusted}
    Identical streams of trusted inputs: for all~$t$, the user messages and
    system policies are the same in both executions.
  \item \label{ass:untrusted}
    Possibly differing untrusted input streams: the sequences
    $U_t^{(1)}$ and $U_t^{(2)}$ of web-retrieved content, document content
    and skill outputs may differ arbitrarily.
  \item \label{ass:verifier}
    The verifier enforces invariants~\ref{inv:taint-dep}
    and~\ref{inv:cp-authority} at every step.
\end{enumerate}

Then for all time steps~$t \geq 0$:
\begin{enumerate}[label=(\roman*)]
  \item \textbf{Action noninterference:}\quad
    $a_t^{(1)} = a_t^{(2)}$.
  \item \textbf{Control-plane noninterference:}\quad
    $\Pt^{(1)} = \Pt^{(2)}$.
  \item \textbf{Tool-output equivalence} (derived):\quad
    Since actions are identical, tool outputs are identical.
\end{enumerate}
\end{theorem}

In words: adversarial variations in untrusted data (web pages, documents or
skill outputs) do not interfere with the agent system's action-selection
function output or control-plane state.

\begin{remark}[On assumption~\ref{ass:trusted} and circularity]
In a na\"{\i}ve formulation, one might assume ``tool results are identical''
as a precondition.  This would be circular: tool results depend on which tools
were called, and we are trying to prove that tool calls are identical.  Our
formulation resolves this:
\begin{enumerate}
  \item \ref{ass:trusted} states only that \emph{user messages and system
    policies} are identical---genuinely exogenous inputs.
  \item Tool-output equivalence is \emph{derived}: since
    $a_t^{(1)} = a_t^{(2)}$ (proved), and both executions call the same tool
    with the same arguments against the same external state, the tool outputs
    at step $t{+}1$ are identical.
  \item This makes the induction well-founded: at step $t{+}1$, tool outputs
    from step~$t$ are identical because the actions at step~$t$ were identical
    (by the inductive hypothesis).
\end{enumerate}
\end{remark}

% ------------------------------------------------------------------
\subsection{Proof}\label{sec:proof}

We proceed by strong induction on the time step~$t$.

\paragraph{Base case ($t = 0$).}
By assumption~\ref{ass:init}, $S_0^{(1)} = S_0^{(2)}$, so in particular
$P_0^{(1)} = P_0^{(2)}$.  The trusted inputs at step~$0$ are identical by
assumption~\ref{ass:trusted}.  We must show $a_0^{(1)} = a_0^{(2)}$.

The controller constructs the IR graph~$G_0$.  By
invariant~\ref{inv:taint-dep}, the action-selection function receives as input
only the projection $\pi_0(G_0) = \{v \in V_0 : \taint(v) = 0\}$.

\begin{lemma}[Untrusted isolation]\label{lem:isolation}
Every IR node derived solely from untrusted input~$U_t$ satisfies
$\taint(v) = 1$.  Conversely, every node with $\taint(v) = 0$ is determined
entirely by the initial state~$S_0$ and the trusted input stream up to
step~$t$.
\end{lemma}

\begin{proof}
By Definition~\ref{def:taint}, raw spans from $\WEB$, $\SKILL$, or
$\TOOLunauth$ have $\taint = 1$.  By Definition~\ref{def:taint-prop}, taint
propagates through the dependency relation: any node depending on a tainted
node is itself tainted.  The only mechanism to clear taint is the
$\textsf{VerifiedFact}$ promotion (Definition~\ref{def:verification}), which
requires verification against trusted sources---a procedure whose output
depends only on trusted data (the content of~$v$, the corroborating trusted
source, the schema, and the allowlist, all of which are functions of trusted
inputs).  Therefore, untrusted input can only produce tainted nodes, and
untainted nodes are functions of trusted data alone.
\end{proof}

Applying Lemma~\ref{lem:isolation}: since the action-selection function
receives only $\pi_0(G_0)$ (untainted nodes), and all untainted nodes at
step~$0$ are identical across both executions (same $S_0$, same trusted input),
the function receives identical input and produces identical output:
$\hat{a}_0^{(1)} = \hat{a}_0^{(2)}$.

The verifier is deterministic and receives the same proposed transition in
both executions (same $S_0$, same $\hat{a}_0$).  Hence
$V^{(1)} = V^{(2)}$, and the same branch of Equation~\eqref{eq:repair} is
taken.  In either case:
\begin{itemize}
  \item If $V = 1$: $a_0^{(1)} = \hat{a}_0^{(1)} = \hat{a}_0^{(2)} = a_0^{(2)}$.
  \item If $V = 0$: $a_0^{(1)} = \varnothing = a_0^{(2)}$.
\end{itemize}

For the control plane: by invariant~\ref{inv:cp-authority}, modifications to
$\Pt$ require principal $p \succeq \USER$.  Untrusted inputs have principal
$\WEB$, $\SKILL$, or $\TOOLunauth$, all of which satisfy
$\WEB, \SKILL \preceq \TOOLunauth \prec \TOOLauth \preceq \USER$.
Therefore, untrusted inputs cannot modify $\Pt$.  Combined with identical
trusted inputs and the same verifier outcome, we get
$P_1^{(1)} = P_1^{(2)}$.

\paragraph{Inductive step.}
Assume that for all $s \leq t$:
\[
  P_s^{(1)} = P_s^{(2)}, \qquad
  a_s^{(1)} = a_s^{(2)}, \qquad
  S_s^{(1)}\big|_{\taint=0} = S_s^{(2)}\big|_{\taint=0},
\]
where $S_s\big|_{\taint=0}$ denotes the restriction of the state to untainted
components (control plane, verified memory, risk budget and untainted IR
nodes).

At step $t+1$:

\begin{enumerate}
  \item \textbf{Identical untainted state.}
    By the induction hypothesis, the untainted portions of $S_t$ are identical
    across both executions.

  \item \textbf{Identical tool results (derived, not assumed).}
    Since $a_t^{(1)} = a_t^{(2)}$ (induction hypothesis), both executions
    call the same tool with the same arguments.  Assuming deterministic tool
    behaviour against the same external state, the tool outputs at step $t+1$
    are identical.

  \item \textbf{IR construction.}
    The controller $T_\theta$ processes $S_t$ and $I_{t+1}$.  Untrusted
    input $U_{t+1}$ may differ, but by Lemma~\ref{lem:isolation} it produces
    only tainted IR nodes.  All untainted nodes are determined by the
    (identical) untainted state, identical trusted inputs, and identical tool
    outputs.

  \item \textbf{Candidate action.}
    By invariant~\ref{inv:taint-dep}, the action-selection function receives
    only the untainted projection $\pi_0(G_{t+1})$, which is identical across
    both executions.  Hence
    $\hat{a}_{t+1}^{(1)} = \hat{a}_{t+1}^{(2)}$.

  \item \textbf{Verifier outcome.}
    The verifier is deterministic and receives the same proposed transition.
    Hence $V^{(1)} = V^{(2)}$, and by the same case analysis as the base
    case, $a_{t+1}^{(1)} = a_{t+1}^{(2)}$.

  \item \textbf{Control-plane update.}
    By invariant~\ref{inv:cp-authority}, only $\SYS$ or $\USER$ principals
    may modify $P_{t+1}$.  Untrusted data cannot alter $P_{t+1}$.  With
    identical trusted inputs and identical verifier outcomes:
    $P_{t+1}^{(1)} = P_{t+1}^{(2)}$.

  \item \textbf{Untainted state propagation.}
    Since the control plane, verified memory (updated only via trusted
    verification), and risk budget (decremented by identical actions) are all
    identical, we have
    $S_{t+1}^{(1)}\big|_{\taint=0} = S_{t+1}^{(2)}\big|_{\taint=0}$.
\end{enumerate}

This completes the induction.

\begin{corollary}[Output noninterference]\label{cor:output}
Under the same assumptions, the agent's user-visible output stream
$(y_0, y_1, \ldots)$ is identical across both executions, provided the output
generation depends only on untainted IR nodes and the control-plane state.
\end{corollary}

\begin{proof}
By Theorem~\ref{thm:noninterference}, $\Pt^{(1)} = \Pt^{(2)}$ and the
untainted IR nodes are identical.  If $y_t$ depends only on these, then
$y_t^{(1)} = y_t^{(2)}$.
\end{proof}

\begin{corollary}[Risk budget invariance]\label{cor:budget}
Under the same assumptions, $B_t^{(1)} = B_t^{(2)}$ for all~$t$.
\end{corollary}

\begin{proof}
The risk budget is decremented by the cost of each tool call.  Since
$a_t^{(1)} = a_t^{(2)}$ for all~$t$, the same costs are incurred, and
$B_t^{(1)} = B_t^{(2)}$ follows by induction.
\end{proof}


% ==================================================================
\section{Discussion}\label{sec:discussion}

\subsection{Scope: True by Construction}

A skeptical reviewer will correctly note that the theorem can be restated as:
\emph{the output of a function is invariant to inputs you delete before
calling the function}.  This is true, and we do not dispute it.  The
scientific contribution is not that this property is surprising, but that:
\begin{enumerate}
  \item It identifies noninterference from IFC as the correct formal
        framework for prompt injection security.
  \item The IR graph, taint tracking, trust lattice, and verifier invariants
        constitute a concrete \emph{system specification}, not just a design
        principle.
  \item It makes the security guarantee precise: under these invariants, we
        get exactly these properties, with exactly these assumptions.
  \item It enables formal analysis of failure modes: when invariants are
        violated, we know exactly which security properties break.
\end{enumerate}

The analogy is to memory safety: ``a program with no buffer overflows has no
buffer-overflow exploits'' is true by construction, but memory-safe languages
are still valuable because they enforce the property systematically.

\subsection{The Utility Gap}

Many tasks require the agent to \emph{read} untrusted content to decide
actions (e.g.\ ``summarize this email, then reply'').  If the action-selection
function cannot condition on untrusted content, the agent is safe but
potentially useless.  Our proposed resolution:
\begin{enumerate}
  \item The model may freely read untrusted content to produce \emph{tainted
        intermediate artifacts} (summaries, extractions).
  \item These artifacts remain tainted ($\taint = 1$) and are stored as
        \textsf{CandidateFact} nodes.
  \item Before influencing action selection, artifacts must pass through
        \textsf{VerifiedFact} promotion (cross-reference, schema validation,
        or user confirmation).
  \item Actions may only depend on verified artifacts ($\taint = 0$).
\end{enumerate}
We have not empirically evaluated this declassification pathway.  Measuring
task completion rate under defense vs.\ baseline is the critical open problem.

\subsection{Threat Model Boundaries}

The theorem does \emph{not} protect against: (a)~malicious user instructions
(the user has \USER-level authority); (b)~compromised system prompts;
(c)~adversary-controlled tools treated as \TOOLauth; (d)~memory poisoning via
false-accept bugs in verification; (e)~adaptive attacks targeting the
verification pathway.

\subsection{Limitations}

\begin{enumerate}
  \item \textbf{Model scale.}  We evaluate with FLAN-T5-base (248M params) as
        a surrogate decision-point model, not a full tool-calling agent.
  \item \textbf{Deterministic verifier assumption.}  The theorem assumes the
        verifier perfectly enforces all invariants.  Implementation bugs could
        violate preconditions.
  \item \textbf{Stochastic decoding.}  We use greedy decoding (deterministic).
        Stochastic sampling requires extension to probabilistic
        noninterference.
  \item \textbf{No utility evaluation.}  We do not measure task completion
        under defense.
  \item \textbf{No declassification evaluation.}  \textsf{VerifiedFact}
        promotion is specified but not empirically tested.
  \item \textbf{Surrogate, not end-to-end.}  We test the decision point, not
        the full agent stack with tool execution.
\end{enumerate}

% ==================================================================
\section{Empirical Validation}\label{sec:eval}

We validate the theorem empirically using real neural network inference on two
canonical benchmarks.

\paragraph{Benchmarks.}
InjecAgent~\cite{zhan2024injecagent} provides 1,054 unique test cases
(510~Direct Harm + 544~Data Stealing) across 17~user tools and 62~attacker
tools, each in base and enhanced attack settings.
BIPIA~\cite{yi2023bipia} spans 5~application scenarios with 49~unique attack
goals (each with 5~variants), yielding up to 15,000 context-attack pairs via
Cartesian product.  We evaluate 400 InjecAgent cases (100/split) and
300~BIPIA cases (100/task type).

\paragraph{Model and scope.}
We use FLAN-T5-base~\cite{chung2022scaling} (248M~params) with deterministic
greedy decoding.  FLAN-T5 is an instruction-following seq2seq model used as a
\emph{surrogate for the action-selection decision point}, not a tool-calling
agent in the ReAct sense.  This tests whether filtering untrusted content from
the decision-point input prevents influence on the output, but is not an
end-to-end agent evaluation.

\paragraph{Methodology: differential testing.}
For each test case, we run four model inferences: baseline with clean input,
baseline with injection, guarded with clean input, and guarded with injection
(taint-stripped).  We measure \emph{influence rate} (whether the output changed
at all due to injection), which directly tests the noninterference property and
is more conservative than traditional ASR.

\paragraph{Results.}
\begin{center}
\begin{tabular}{lcc}
\toprule
\textbf{Metric} & \textbf{InjecAgent} ($n\!=\!400$) & \textbf{BIPIA} ($n\!=\!300$) \\
\midrule
Baseline influence rate & 21.8\% & 6.0\% \\
Guarded influence rate  & 0.0\%  & 0.0\% \\
Noninterference rate    & 100\%  & 100\% \\
\bottomrule
\end{tabular}
\end{center}

The 0\% guarded influence rate is \emph{expected by construction}: the guarded
agent receives identical input regardless of whether an injection is present,
so under deterministic decoding, identical input produces identical output.
The result confirms the enforcement layer is correctly implemented; any result
above 0\% would indicate an implementation bug.

% ==================================================================
\section{Conclusion}\label{sec:conclusion}

Indirect prompt injection collapses the boundary between data and control by
hiding malicious instructions in external content~\cite{crowdstrike2025}.
The noninterference theorem shows that a system-level enforcement
layer---typed IR, taint tracking, and authority-based verification---can
restore this boundary: the action-selection function's output is invariant to
arbitrary adversarial variations in untrusted input, provided tainted content
is excluded from its input.

The theorem is, by design, a formalisation of a filtering rule---and we
believe this is its strength, not its weakness.  It provides a precise
specification of what must be built, what assumptions it rests on, and what
breaks when those assumptions are violated.  The critical open problem is the
\emph{utility gap}: demonstrating that the declassification pathway preserves
task capability while maintaining the security guarantee.

% ==================================================================
\bibliographystyle{plainnat}
\bibliography{references}

\end{document}
